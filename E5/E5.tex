%%%%%%%%%%%%%%%%%%%%%%%%%%%%%%%%%%%%%%%%%%%%%%%%%%%%%%%%%%%%%%%%%%%%%%%%%%%%%%%%%%%%
%Do not alter this block of commands.  If you're proficient at LaTeX, you may include additional packages, create macros, etc. immediately below this block of commands, but make sure to NOT alter the header, margin, and comment settings here. 
\documentclass[12pt]{article}
 \usepackage[margin=1in]{geometry} 
\usepackage{amsmath,amsthm,amssymb,amsfonts, enumitem, fancyhdr, color, comment, graphicx, environ,array}
\usepackage[labelformat=empty]{caption}
\pagestyle{fancy}
\setlength{\headheight}{65pt}
\newenvironment{problem}[2][Problema]{\begin{trivlist}
\item[\hskip \labelsep {\bfseries #1}\hskip \labelsep {\bfseries #2.}]}{\end{trivlist}}
\newenvironment{sol}
    {\emph{Solución:}
    }
    {
    }
\specialcomment{com}{ \color{blue} \textbf{Comment:} }{\color{black}} %for instructor comments while grading
\NewEnviron{probscore}{\marginpar{ \color{blue} \tiny Problem Score: \BODY \color{black} }}
%%%%%%%%%%%%%%%%%%%%%%%%%%%%%%%%%%%%%%%%%%%%%%%%%%%%%%%%%%%%%%%%%%%%%%%%%%%%%%%%%





%%%%%%%%%%%%%%%%%%%%%%%%%%%%%%%%%%%%%%%%%%%%%
%Fill in the appropriate information below
\lhead{Héctor Alejandro Márquez González}  %replace with your name
\rhead{Grupo 002 \\ 1989936} %replace XYZ with the homework course number, semester (e.g. ``Spring 2019"), and assignment number.
%%%%%%%%%%%%%%%%%%%%%%%%%%%%%%%%%%%%%%%%%%%%%


%%%%%%%%%%%%%%%%%%%%%%%%%%%%%%%%%%%%%%
%Do not alter this block.
\begin{document}
%%%%%%%%%%%%%%%%%%%%%%%%%%%%%%%%%%%%%%
Se realizó un muestreo para comparar el nivel de glucosa en la sangre de pacientes de un hospital cuyo seguimiento está a cargo de diferente médico. Use $\alpha=0.05$. Los resultados en la siguiente tabla:

\begin{table}[h!]
\centering
\begin{tabular}{|c|c|c|c|}
\hline
\textbf{Dr. Álvarez} & \textbf{Dr. Benítez} & \textbf{Dr. Casas} & \textbf{Dr. Domínguez} \\ \hline
72 & 112 & 73 & 119 \\ \hline
92 & 123 & 70 & 76 \\ \hline
108 & 122 & 117 & 94 \\ \hline
97 & 122 & 114 & 74 \\ \hline
72 & 112 & 111 & 107 \\ \hline
69 & 85 & 109 & 115 \\ \hline
117 & 98 & 92 & 66 \\ \hline
87 & 105 & 89 & 94 \\ \hline
94 & 110 & 80 & 122 \\ \hline
109 & 112 & 111 & 68 \\ \hline
94 & 69 & 122 & 95 \\ \hline
96 & 86 & 69 & 88 \\ \hline
112 & 78 & 72 & 82 \\ \hline
80 & 117 & 90 & 114 \\ \hline
100 & 114 & 69 & 109 \\ \hline
81 & 84 & 87 & 105 \\ \hline
86 & 74 & 111 & 97 \\ \hline
113 & 103 & 85 & 95 \\ \hline
107 & 109 & 82 & 88 \\ \hline
77 & 100 & 70 & 101 \\ \hline
84 & & 114 & \\ \hline
99 & & 110 & \\ \hline
73 & & 116 & \\ \hline
69 & & 124 & \\ \hline
103 & & 72 & \\ \hline
103 & & 120 & \\ \hline
117 & & 106 & \\ \hline
100 & & & \\ \hline
66 & & & \\ \hline
66 & & & \\ \hline
84 & & & \\ \hline
\end{tabular}
\label{tab:glucosa}
\end{table}

\pagebreak
%-----------------------------------------------------------------------------------
%-----------------------------------------------------------------------------------
%-----------------------------------------------------------------------------------
%-----------------------------------------------------------------------------------
%-----------------------------------------------------------------------------------
%-----------------------------------------------------------------------------------
%--------------------------------------------A--------------------------------------
\begin{problem}{A} 
Se desea averiguar si el nivel promedio de glucosa en los pacientes del doctor Álvarez es considerado controlado. Si un nivel considerado normal es de entre 80 y 130 mg/dl, alto valores mayores y bajo para valores menores, controlado sería que no exceda de 130 mg/dl.
\end{problem}

\begin{sol}
\begin{table}[h!]
\centering
\begin{tabular}{|>{\raggedright\arraybackslash}m{6cm}|>{\raggedright\arraybackslash}m{6cm}|}
\hline
 & \textbf{Prueba de Hipótesis} \\ \hline
$H_0$ & $\mu=130$ \\ \hline
$H_a$ & $\mu\neq130$ \\ \hline
\textbf{Estadístico de Prueba} & \vspace{0.5cm}$t=\frac{\bar{x}-\mu}{\frac{S}{\sqrt{n}}}=-13.612$\vspace{0.5cm} \\ \hline
\textbf{p valor} & $2.248e-14$ \\ \hline
\textbf{Comparativa y conclusión respecto al rechazo de $H_0$} & Debido a que p-valor$=2.248e-14<0.05$ entonces hay suficiente evidencia para rechazar $H_0$ y aceptamos la hipótesis alternativa $H_a: \mu\neq130$ \\ \hline
\textbf{Intervalo de confianza} & $(85.37118,97.01592)$ \\ \hline
\textbf{Conclusión en contexto del ejercicio} & Debido a que rechazamos la hipotesis nula y tenemos el intervalo de 95\% de confianza de $(85.37118,97.01592)$ se concluye que el nivel promedio de glucosa en los pacientes del doctor Álvarez es controlado, esto es, que está entre 80 y 130 mg/dl. \\ \hline
\end{tabular}
\label{tab:hipotesis}
\end{table}

\begin{figure}[h]  % 'h' significa que la figura se colocará aquí
    \centering      % Centra la imagen
    \includegraphics[width=0.5\textwidth]{a.png} 
    \caption{Código en R del ejercicio A}
\end{figure}

\end{sol}

\pagebreak
%-----------------------------------------------------------------------------------
%-----------------------------------------------------------------------------------
%-----------------------------------------------------------------------------------
%-----------------------------------------------------------------------------------
%-----------------------------------------------------------------------------------
%-----------------------------------------------------------------------------------
%--------------------------------------------B--------------------------------------
\begin{problem}{B}
Entre los doctores, es sabido que algunos consideran complementar los medicamentos con algunos remedios herbolarios o tés. Se desea averiguar si el nivel promedio de glucosa en los pacientes del doctor Casas es considerado mayor que el nivel de glucosa promedio en pacientes del doctor Domínguez, ya que éste último suele recomendar complementar el medicamento con dicho tipo de remedios.  Según la evidencia ¿suele ser mejor el complementar el medicamento con remedios naturales?

\end{problem}
\begin{sol}
Primero haremos una prueba de hipótesis para ver si las varianzas son iguales, esto es:

\begin{table}[h!]
\centering
\begin{tabular}{|>{\raggedright\arraybackslash}m{6cm}|>{\raggedright\arraybackslash}m{6cm}|}
\hline
 & \textbf{Prueba de Hipótesis} \\ \hline
$H_0$ & \vspace{0.5cm}$\frac{\sigma_{\text{Casas}}^2}{\sigma_{\text{Dominguez}
}^2}=1$ \vspace{0.5cm}\\ \hline
$H_a$ & \vspace{0.5cm} $\frac{\sigma_{\text{Casas}}^2}{\sigma_{\text{Dominguez}
}^2}\neq1$ \vspace{0.5cm}\\ \hline
\textbf{Estadístico de Prueba} & \vspace{0.5cm}$F = \frac{s_\text{Casas}^2}{s_\text{Dominguez}^2}=1.1213$\vspace{0.5cm} \\ \hline
\textbf{p valor} & $0.7732$ \\ \hline
\textbf{Comparativa y conclusión respecto al rechazo de $H_0$} & Debido a que p-valor$=0.7732>0.05$ entonces hay suficiente evidencia para no rechazar $H_0$. \\ \hline
\textbf{Intervalo de confianza} & $(0.4888093,2.7247331)$ \\ \hline
\textbf{Conclusión en contexto del ejercicio} & Se concluye que no hay suficiente evidencia para decir que las varianzas son diferentes, por tanto, para analizar la eficacia de remedios naturales entre los pacientes de los doctores Casas y Dominguez usaremos nuestro estadístico de prueba $t$ donde $n_1 ,n_2 < 30$ y varianzas poblacionales desconocidas pero iguales \\ \hline
\end{tabular}
\label{tab:hipotesis}
\end{table}
\pagebreak

\begin{figure}[h]  % 'h' significa que la figura se colocará aquí
    \centering      % Centra la imagen
    \includegraphics[width=0.8\textwidth]{b1.png} 
    \caption{Código en R del ejercicio B para las varianzas}
\end{figure}

Ahora haremos la prueba de hipótesis para las medias, donde las varianzas son desconocidas pero iguales.
\begin{table}[h!]
\centering
\begin{tabular}{|>{\raggedright\arraybackslash}m{6cm}|>{\raggedright\arraybackslash}m{6cm}|}
\hline
 & \textbf{Prueba de Hipótesis} \\ \hline
$H_0$ & $\mu_{\text{Casas}}-\mu_{\text{Dominguez}}=0$ \\ \hline
$H_a$ & $\mu_{\text{Casas}}-\mu_{\text{Dominguez}}>0$ \\ \hline
\textbf{Estadístico de Prueba} & \vspace{0.5cm}$t=\frac{\bar{x_1}-\bar{x_2}}{S_p\sqrt{\frac{1}{n_1}+\frac{1}{n_2}}}=-1.4735$\vspace{0.5cm} \\ \hline
\textbf{p valor} & $0.9262$ \\ \hline
\textbf{Comparativa y conclusión respecto al rechazo de $H_0$} & Debido a que p-valor$=0.8707>0.05$ entonces hay suficiente evidencia para no rechazar $H_0$. \\ \hline
\textbf{Intervalo de confianza} & $(-16.6385, \infty)$ \\ \hline
\textbf{Conclusión en contexto del ejercicio} & Se concluye que no hay suficiente evidencia para decir que suele ser mejor el complementar el medicamento con remedios naturales. \\ \hline
\end{tabular}
\label{tab:hipotesis}
\end{table}
\pagebreak
\begin{figure}[h]  % 'h' significa que la figura se colocará aquí
    \centering      % Centra la imagen
    \includegraphics[width=0.8\textwidth]{b2.png} 
    \caption{Código en R del ejercicio B para las medias}
\end{figure}
\end{sol}
\pagebreak
%-----------------------------------------------------------------------------------
%-----------------------------------------------------------------------------------
%-----------------------------------------------------------------------------------
%-----------------------------------------------------------------------------------
%-----------------------------------------------------------------------------------
%-----------------------------------------------------------------------------------
%--------------------------------------------C--------------------------------------
\begin{problem}{C}
El jefe de departamento desea analizar si la proporción de pacientes “no controlados” del doctor Domínguez (valores mayores a lo considerado normal) es mayor al 15\%, ya que de ser así se buscaría recomendar al doctor tomar un curso de actualización. Según la evidencia ¿es conveniente hacer la recomendación al doctor Domínguez?
\end{problem}

\begin{sol}
En este caso $p_o=0.15$ y $n=27$ entonces se cumple que $np_o<10$ por lo que no podemos usar una aproximación normal, entonces usaremos la binomial:
\begin{table}[h!]
\centering
\begin{tabular}{|>{\raggedright\arraybackslash}m{6cm}|>{\raggedright\arraybackslash}m{6cm}|}
\hline
 & \textbf{Prueba de Hipótesis} \\ \hline
$H_0$ & $p=0.15$ \\ \hline
$H_a$ & $p>0.15$ \\ \hline
\textbf{Estadístico de Prueba} & \vspace{0.5cm}$X=5$\vspace{0.5cm} \\ \hline
\textbf{p valor} & $0.3813$ \\ \hline
\textbf{Comparativa y conclusión respecto al rechazo de $H_0$} & Debido a que p-valor$=0.3813>0.05$ entonces hay suficiente evidencia para no rechazar $H_0$. \\ \hline
\textbf{Intervalo de confianza} & $(0.0759362,1.0000000)$ \\ \hline
\textbf{Conclusión en contexto del ejercicio} & Se concluye que no hay suficiente evidencia para decir que es conveniente hacer la recomendación al doctor dominguez, pues no tenemos suficiente evidencia para decir que la proporción de pacientes no controlados del Dr. Domínguez es mayor que el 15\% \\ \hline
\end{tabular}
\label{tab:hipotesis}
\end{table}

\begin{figure}[h]  % 'h' significa que la figura se colocará aquí
    \centering      % Centra la imagen
    \includegraphics[width=0.8\textwidth]{c.png} 
    \caption{Código en R del ejercicio C}
\end{figure}
\end{sol}
\pagebreak

%%%%%%%%%%%%%%%%%%%%%%%%%%%%%%%%%%%%%
%Do not alter anything below this line.
\end{document}
